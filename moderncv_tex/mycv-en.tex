%% start of file `template-zh.tex'.
%% Copyright 2006-2012 Xavier Danaux (xdanaux@gmail.com).
%
% This work may be distributed and/or modified under the
% conditions of the LaTeX Project Public License version 1.3c,
% available at http://www.latex-project.org/lppl/.


\documentclass[11pt,a4paper,sans]{moderncv}   % possible options include font size ('10pt', '11pt' and '12pt'), paper size ('a4paper', 'letterpaper', 'a5paper', 'legalpaper', 'executivepaper' and 'landscape') and font family ('sans' and 'roman')

% moderncv 主题
\moderncvstyle{classic}                        % 选项参数是 ‘casual’, ‘classic’, ‘oldstyle’ 和 ’banking’
\moderncvcolor{blue}                          % 选项参数是 ‘blue’ (默认)、‘orange’、‘green’、‘red’、‘purple’ 和 ‘grey’
%\nopagenumbers{}                             % 消除注释以取消自动页码生成功能

% 字符编码
%\usepackage[utf8]{inputenc}                   % 替换你正在使用的编码
%\usepackage{CJKutf8}
\usepackage[cm-default]{fontspec} %provide font selecting commands
\usepackage{xunicode}  % provide unicode character macros

\defaultfontfeatures{Mapping=tex-text}

%\usepackage[slantfont, boldfont]{xeCJK}
\usepackage{xltxtra} % provide some fixes/extras
%\CJKlanguage{zh-CN}
\usepackage{xcolor}
%\setCJKmainfont{WenQuanYi Zen Hei}
%\setCJKfamilyfont{song}{WenQuanYi Zen Hei}

% 调整页面出血
\usepackage[scale=0.75]{geometry}
%\setlength{\hintscolumnwidth}{3cm}           % 如果你希望改变日期栏的宽度

%\usepackage{hyperref}

% 个人信息
\firstname{ZhenXing}
\familyname{Su}
%\title{简历题目 (可选项)}                      % 可选项、如不需要可删除本行
%\address{街道及门牌号}{邮编及城市}             % 可选项、如不需要可删除本行
\mobile{+86~***********}                         % 可选项、如不需要可删除本行
%\phone{+2~(345)~678~901}                          % 可选项、如不需要可删除本行
%\fax{+3~(456)~789~012}                            % 可选项、如不需要可删除本行
\email{suzp1984@gmail.com}                    % 可选项、如不需要可删除本行
\homepage{zpcat.blogspot.com}                  % 可选项、如不需要可删除本行
%\extrainfo{附加信息 (可选项)}                  % 可选项、如不需要可删除本行
\photo[64pt][0.4pt]{myself-new}                  % ‘64pt’是图片必须压缩至的高度、‘0.4pt‘是图片边框的宽度 (如不需要可调节至0pt)、’picture‘ 是图片文件的名字;可选项、如不需要可删除本行
%\quote{引言(可选项)}                           % 可选项、如不需要可删除本行

% 显示索引号;仅用于在简历中使用了引言
%\makeatletter
%\renewcommand*{\bibliographyitemlabel}{\@biblabel{\arabic{enumiv}}}
%\makeatother

% 分类索引
%\usepackage{multibib}
%\newcites{book,misc}{{Books},{Others}}
%----------------------------------------------------------------------------------
%            内容
%----------------------------------------------------------------------------------
\begin{document}
%\begin{CJK}{UTF8}{gbsn}                       % 详情参阅CJK文件包
\maketitle

\section{Educational Background}
\cventry{2007.9-10.7}{Master}{Institute of Metal Research, Chinese Academy of Sciences}{Shen Yang} {Chemical Physics} {} % {\textit{成绩}}{说明}  % 第3到第6编码可留白
\cventry{2003.9-07.7}{Bachelor}{YanTai university}{YanTai} {Applied Physics} {} % {\textit{成绩}}{说明}

%\section{毕业论文}
%\cvitem{题目}{\emph{题目}}
%\cvitem{导师}{导师}
%\cvitem{说明}{\small 论文简介}

\section{Working Experience}
%\subsection{专业}
\cventry{2014.4-16.4}{Software Engineer}{Iboxpay Shenzhen}{}{}{%
}
\cvitem{Duty:}{%
\begin{itemize}
\item {\small \emph {Android app engineer}}
\item {\small \emph {Software Engineer at Architecture Team}}
\end{itemize}
}
\cventry{2013.9-14.4}{Software Engineer}{Topwise3g Shenzhen}{}{}{}
\cvitem{Duty:}{%
\begin{itemize}
\item {\small \emph {Bluetooth Low Energy Engineer}}
\item {\small \emph {Android framework engineer}}
\end{itemize}
}
\cventry{2012.2-12.8}{Software Engineer}{Topwise Shenzhen}{}{}{%
}
\cvitem{Duty:}{%
\begin{itemize}
\item {\small \emph {Android framework engineer}}
\end{itemize}}
\cventry{2010.4-12.1}{Software Engineer}{Topwise3g Shenzhen}{}{}{}
\cvitem{Duty:}{%
\begin{itemize}
\item {\small \emph {Android framework engineer}}
\end{itemize}} %\newline{}

\section{Programming Skills}
\cvitem{github}{\small \href{https://github.com/suzp1984}{\color{blue}{https://github.com/suzp1984}}}
\cvitem{blog}{\small \href{http://zpcat.blogspot.com/}{\color{blue}{http://zpcat.blogspot.com/}}}
\cvitem{Languages}{\small C, Emacs Lisp, Python, Java, Javascript, Shell, \LaTeX{}}
\cvitem{Operation System} {Linux(expert), Mac OS X(Daily usage)}


\section{OpenSource codes}
\cvitem{\href{https://github.com/suzp1984/Light_BLE}{\color{blue}{Light-BLE}}}{\small BLE(Bluetooth Low Energy) Device debug tool for Android.}
\cvitem{\href{https://github.com/suzp1984/jbig-android}{\color{blue}{jbig-android}}}{\small jbig single-color space picture codec for Android.}

\section{Projects experience}
\subsection{IBoxPay's CashBox Android App Project}
\cventry{2014.7-15.5}{Android App Engineer}{IBoxPay Co. Ltd}{ShenZhen}{}{%
redesign and refactor the CashBox App project.\newline{}%
}
\cvitem{Description:}{%
\begin{itemize}
\item CashBox App is IboxPay's core project, which consists of the mobile client side app, the backend side trading system, and an intelligent terminal hardware. This is what we called the mobile POS, a box hardware must be connected to the mobile side CashBox app, which acts as a middleware between the terminal box and the backend trading system. The customer swipes his bank card in that little portable hardware, then he can pay the bill to the merchants.
\item The intelligent terminals can be classified into its connection channel types, there are at least three types of channels as Audio jack, Bluetooth Classic, Bluetooth Low Energy and UART serial port.
\end{itemize}
}

\cvitem{Achievement:}{%
\begin{itemize}
\item I rewrote the code according to the Object-Oriented Principal, decoupled the code by introduced an isolated android module project which can be reused in another project.
\item I make the development process sustainable and the code is readable by wrote the software design document, wrote necessary unit-test and also maintenance a coding style document.
\item The highlighted part of that code is the hardware connection channel part,  I  introduced a Connection interface, all the connection types, including Audio Jack, Bluetooth was just an implementation of that interface, then, when another project works on a new terminal with UART custom connection channel, what it did is just implement a new Connection interface.
\item I porting the JBIG codec to Java environment by using the JNI method. JBIG codec is an efficient lossless compression algorithm for single color depth space picture.
\end{itemize}}
\cvitem{}{}

\subsection{Jobs at Software Architecture Team of IBoxPay}
\cventry{2015.7-16.3}{Software Developer}{IBoxPay Co. Ltd}{ShenZhen}{}{%
}
\cvitem{Description:}{%
\begin{itemize}
\item Android App Architecture research.
\item Git and Gitlab Training.
\item Apache and Nginx journal report analysis.
\item Nginx Lua module develop(OpenResty).
\end{itemize}
}
\cvitem{Achievement:}{%
\begin{itemize}
\item Abandon the outdated centralized version control system, SVN, use the advanced distributed version control system Git, the company also start to use the popular on-line coding review and authority control web app, Gitlab.
\end{itemize}
}

\cvitem{}{}
\subsection{Bluetooth Low Energy Project}
\cventry{2013.10-14.3}{Software Developer}{Topwise3g Co. Ltd}{Shenzhen}{}{%
}
\cvitem{Description:}{%
\begin{itemize}
\item Research and analysis Bluetooth Low Energy application at Broadcom's BLE board.
\end{itemize}}
\cvitem{Achievement:}{%
\begin{itemize}
\item Open Sourced \href{https://github.com/suzp1984/Light_BLE}{\color{blue}{Light-BLE}} project, which can be used to debug and analysis the peripheral BLE device during development.
\end{itemize}
}
\cvitem{}{}

\subsection{Factory autotest toolkit for SpreadTrum's Android platform}
\cventry{2012.5-12.7}{Software Developer}{Topwise Co. Ltd}{Shenzhen}{}{%
}
\cvitem{Description:}{%
\begin{itemize}
\item The factory auto running test toolkit is running in an autotest machine which checks the newly produced PCB board.
\end{itemize}
}
\cvitem{Achievement:}{%
\begin{itemize}
\item check out the faulty PCB board at the early stage in the factory, then promote the rate of qualified PCB board out of the factory.
\end{itemize}
}
\cvitem{}{}

\subsection{Android Framework Development}
\cventry{2010.5-12.8}{Software Developer}{Topwise3g \& Topwise Co. Ltd}{Shenzhen}{}{}
\cvitem{Description:}{%
\begin{itemize}
\item Android framework and System Developer from version 1.6 to 4.3, My duties include integration Makefile development and HAL layer development.
\end{itemize}}

%\cvdoubleitem{类别 1}{XXX, YYY, ZZZ}{类别 4}{XXX, YYY, ZZZ}
%\cvdoubleitem{类别 2}{XXX, YYY, ZZZ}{类别 5}{XXX, YYY, ZZZ}
%\cvdoubleitem{类别 3}{XXX, YYY, ZZZ}{类别 6}{XXX, YYY, ZZZ}

%\section{个人兴趣}
%\cvitem{骑车}{\small 喜欢骑车旅行}
%\cvitem{爱好 2}{\small 说明}
%\cvitem{爱好 3}{\small 说明}

%\section{其他 1}
%\cvlistitem{项目 1}
%\cvlistitem{项目 2}
%\cvlistitem{项目 3}

\renewcommand{\listitemsymbol}{-}             % 改变列表符号

%\section{其他 2}
%\cvlistdoubleitem{项目 1}{项目 4}
%\cvlistdoubleitem{项目 2}{项目 5\cite{book1}}
%\cvlistdoubleitem{项目 3}{}

% 来自BibTeX文件但不使用multibib包的出版物
%\renewcommand*{\bibliographyitemlabel}{\@biblabel{\arabic{enumiv}}}% BibTeX的数字标签
\nocite{*}
\bibliographystyle{plain}
\bibliography{publications}                    % 'publications' 是BibTeX文件的文件名

% 来自BibTeX文件并使用multibib包的出版物
%\section{出版物}
%\nocitebook{book1,book2}
%\bibliographystylebook{plain}
%\bibliographybook{publications}               % 'publications' 是BibTeX文件的文件名
%\nocitemisc{misc1,misc2,misc3}
%\bibliographystylemisc{plain}
%\bibliographymisc{publications}               % 'publications' 是BibTeX文件的文件名

%\clearpage\end{CJK}
\end{document}


%% 文件结尾 `template-zh.tex'.
