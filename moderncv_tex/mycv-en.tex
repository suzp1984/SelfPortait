%% start of file `template-zh.tex'.
%% Copyright 2006-2012 Xavier Danaux (xdanaux@gmail.com).
%
% This work may be distributed and/or modified under the
% conditions of the LaTeX Project Public License version 1.3c,
% available at http://www.latex-project.org/lppl/.


\documentclass[11pt,a4paper,sans]{moderncv}   % possible options include font size ('10pt', '11pt' and '12pt'), paper size ('a4paper', 'letterpaper', 'a5paper', 'legalpaper', 'executivepaper' and 'landscape') and font family ('sans' and 'roman')

% moderncv 主题
\moderncvstyle{classic}                        % 选项参数是 ‘casual’, ‘classic’, ‘oldstyle’ 和 ’banking’
\moderncvcolor{blue}                          % 选项参数是 ‘blue’ (默认)、‘orange’、‘green’、‘red’、‘purple’ 和 ‘grey’
%\nopagenumbers{}                             % 消除注释以取消自动页码生成功能

% 字符编码
%\usepackage[utf8]{inputenc}                   % 替换你正在使用的编码
%\usepackage{CJKutf8}
\usepackage[cm-default]{fontspec} %provide font selecting commands
\usepackage{xunicode}  % provide unicode character macros

\defaultfontfeatures{Mapping=tex-text}

%\usepackage[slantfont, boldfont]{xeCJK}
\usepackage{xltxtra} % provide some fixes/extras
%\CJKlanguage{zh-CN}
\usepackage{xcolor}
%\setCJKmainfont{WenQuanYi Zen Hei}
%\setCJKfamilyfont{song}{WenQuanYi Zen Hei}

% 调整页面出血
\usepackage[scale=0.75]{geometry}
%\setlength{\hintscolumnwidth}{3cm}           % 如果你希望改变日期栏的宽度

%\usepackage{hyperref}

% 个人信息
\firstname{ZhenXing}
\familyname{Su}
%\title{简历题目 (可选项)}                      % 可选项、如不需要可删除本行
%\address{街道及门牌号}{邮编及城市}             % 可选项、如不需要可删除本行
\mobile{+86~18589090105}                         % 可选项、如不需要可删除本行
%\phone{+2~(345)~678~901}                          % 可选项、如不需要可删除本行
%\fax{+3~(456)~789~012}                            % 可选项、如不需要可删除本行
\email{suzp1984@gmail.com}                    % 可选项、如不需要可删除本行
\homepage{zpcat.blogspot.com}                  % 可选项、如不需要可删除本行
%\extrainfo{附加信息 (可选项)}                  % 可选项、如不需要可删除本行
\photo[64pt][0.4pt]{myself-new}                  % ‘64pt’是图片必须压缩至的高度、‘0.4pt‘是图片边框的宽度 (如不需要可调节至0pt)、’picture‘ 是图片文件的名字;可选项、如不需要可删除本行
%\quote{引言(可选项)}                           % 可选项、如不需要可删除本行

% 显示索引号;仅用于在简历中使用了引言
%\makeatletter
%\renewcommand*{\bibliographyitemlabel}{\@biblabel{\arabic{enumiv}}}
%\makeatother

% 分类索引
%\usepackage{multibib}
%\newcites{book,misc}{{Books},{Others}}
%----------------------------------------------------------------------------------
%            内容
%----------------------------------------------------------------------------------
\begin{document}
%\begin{CJK}{UTF8}{gbsn}                       % 详情参阅CJK文件包
\maketitle

\section{Educational Background}
\cventry{2007.8-10.7}{Master}{Institute of Metal Research, Chinese Academy of Sciences}{Shen Yang} {Physics} {} % {\textit{成绩}}{说明}  % 第3到第6编码可留白
\cventry{2003.8-07.7}{Bachelor}{YanTai university}{YanTai} {Physics} {} % {\textit{成绩}}{说明}

%\section{毕业论文}
%\cvitem{题目}{\emph{题目}}
%\cvitem{导师}{导师}
%\cvitem{说明}{\small 论文简介}

\section{Working Experience}
%\subsection{专业}
\cventry{2014.5-16.3}{Software Engineer}{Iboxpay Shenzhen}{}{}{Android app developer %\newline{}
}

\cventry{2010.4-14.4}{Software Engineer}{Topwise Shenzhen}{}{}{As a System engineer of Android %\newline{}
% 工作内容:
% \begin{itemize}
% \item android编译环境开发与维护;
% \item WIFI子系统维护;
% \item Bluetooth子系统维护;
% \item 传感器子系统开发;
% \item 客户定制系统的开发工作;
% \item 工厂测试和自动测试软件开发;
% \item android app 开发;
% \item Bluetooth BLE 协议开发
% \end{itemize}
}

%\cventry{年 -- 年}{职位}{公司}{城市}{}{说明行1\newline{}说明行2}
%\subsection{其他}
%\cventry{年 -- 年}{职位}{公司}{城市}{}{说明}

%\section{语言技能}
%\cvitemwithcomment{英语}{四级}{}
%\cvitemwithcomment{语言 3}{水平}{评价}

\section{Programming Skills}
\cvitem{github}{\small \href{https://github.com/suzp1984}{\color{blue}{https://github.com/suzp1984}}}
\cvitem{blog}{\small \href{http://zpcat.blogspot.com/}{\color{blue}{http://zpcat.blogspot.com/}}}
\cvitem{Languages}{\small C, Emacs Lisp, Python, Java, javascript, node.js}
\cvitem{Operation System} {Linux(expert), Mac OS X(Daily usage)}
%\cvitem{Networking programming} {Twisted, node.js}
%\cvitem{DataBase programming} {}

\section{OpenSource codes}
\cvitem{\href{https://github.com/suzp1984/Light_BLE}{\color{blue}{Light-BLE}}}{\small BLE(Bluetooth Low Energy) Device debug tool for Android.}
\cvitem{\href{https://github.com/suzp1984/jbig-android}{\color{blue}{jbig-android}}}{\small jbig single-color space picture codec for Android.}
\cvitem{\href{https://github.com/suzp1984/TLSDemo}{\color{blue}{TLSDemo}}}{\small TLSDemo is an android app sample, which illustrate how to do https request both in self-signed certificate and two-way SSL client-side verfication.}
\cvitem{\href{https://github.com/suzp1984/tts-mode}{\color{blue}{TTS-mode}}}{\small TTS(Text to speech) mode for emacs.}
\cvitem{\href{https://github.com/suzp1984/fs-mode}{\color{blue}{fs-mode}}}{\small Linux file system management mode for emacs.}
\cvitem{\href{https://github.com/suzp1984/pulseaudio-mode}{\color{blue}{pulseaudio-mode}}}{\small Pulseaudio mode for emacs.}
\cvitem{\href{https://github.com/suzp1984/genesis}{\color{blue}{genesis}}}{\small a plugin desgined middle ware, implemented the reactor module}

%\cvitem{C}{\small 工作中主要使用的语言, 对它也最熟悉}
%\cvitem{java}{\small 开发android工作中主要使用的语言}
%\cvitem{common lisp}{\small 个人兴趣主要钻研的语言, 使用的是sbcl的实现}
%\cvitem{python}{\small 兴趣}
%\cvitem{linux}{\small 使用的操纵系统,个人兴趣喜欢研究它, 本人对程序感兴趣就是从学习使用linux开始的}
%\cvitem{linux kernel}{\small 略懂}
%\cvitem{emacs}{\small 主要的开发工具}
%\cvitem{latex}{\small 正式的文档工具}
%\cvitem{虚拟机}{\small qemu jvm}
%\cvitem{加密技术}{\small 个人爱好}

\section{Projects experience}
\cvitem{1.}{Virtualization interest, xen/kvm, docker, vagrant}
\cvitem{2.}{Android App development}
\cvitem{3.}{Bluetooth v4.0 Low Energy host Stack development, expert of bluedroid, bluez stack.}
\cvitem{4.}{Android System developement}
\cvitem{5.}{Web front-end framework: bootstrap and polymer; and JS MVC framework: angularJs,reactJs.}
%\cvdoubleitem{类别 1}{XXX, YYY, ZZZ}{类别 4}{XXX, YYY, ZZZ}
%\cvdoubleitem{类别 2}{XXX, YYY, ZZZ}{类别 5}{XXX, YYY, ZZZ}
%\cvdoubleitem{类别 3}{XXX, YYY, ZZZ}{类别 6}{XXX, YYY, ZZZ}

%\section{个人兴趣}
%\cvitem{骑车}{\small 喜欢骑车旅行}
%\cvitem{爱好 2}{\small 说明}
%\cvitem{爱好 3}{\small 说明}

%\section{其他 1}
%\cvlistitem{项目 1}
%\cvlistitem{项目 2}
%\cvlistitem{项目 3}

\renewcommand{\listitemsymbol}{-}             % 改变列表符号

%\section{其他 2}
%\cvlistdoubleitem{项目 1}{项目 4}
%\cvlistdoubleitem{项目 2}{项目 5\cite{book1}}
%\cvlistdoubleitem{项目 3}{}

% 来自BibTeX文件但不使用multibib包的出版物
%\renewcommand*{\bibliographyitemlabel}{\@biblabel{\arabic{enumiv}}}% BibTeX的数字标签
\nocite{*}
\bibliographystyle{plain}
\bibliography{publications}                    % 'publications' 是BibTeX文件的文件名

% 来自BibTeX文件并使用multibib包的出版物
%\section{出版物}
%\nocitebook{book1,book2}
%\bibliographystylebook{plain}
%\bibliographybook{publications}               % 'publications' 是BibTeX文件的文件名
%\nocitemisc{misc1,misc2,misc3}
%\bibliographystylemisc{plain}
%\bibliographymisc{publications}               % 'publications' 是BibTeX文件的文件名

%\clearpage\end{CJK}
\end{document}


%% 文件结尾 `template-zh.tex'.
