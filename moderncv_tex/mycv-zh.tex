%% start of file `template-zh.tex'.
%% Copyright 2006-2012 Xavier Danaux (xdanaux@gmail.com).
%
% This work may be distributed and/or modified under the
% conditions of the LaTeX Project Public License version 1.3c,
% available at http://www.latex-project.org/lppl/.


\documentclass[11pt,a4paper,sans]{moderncv}   % possible options include font size ('10pt', '11pt' and '12pt'), paper size ('a4paper', 'letterpaper', 'a5paper', 'legalpaper', 'executivepaper' and 'landscape') and font family ('sans' and 'roman')

% moderncv 主题
\moderncvstyle{classic}                        % 选项参数是 ‘casual’, ‘classic’, ‘oldstyle’ 和 ’banking’
\moderncvcolor{blue}                          % 选项参数是 ‘blue’ (默认)、‘orange’、‘green’、‘red’、‘purple’ 和 ‘grey’
%\nopagenumbers{}                             % 消除注释以取消自动页码生成功能

% 字符编码
%\usepackage[utf8]{inputenc}                   % 替换你正在使用的编码
%\usepackage{CJKutf8}
\usepackage[cm-default]{fontspec} %provide font selecting commands
\usepackage{xunicode}  % provide unicode character macros

\defaultfontfeatures{Mapping=tex-text}

\usepackage[slantfont, boldfont]{xeCJK}
\usepackage{xltxtra} % provide some fixes/extras
%\CJKlanguage{zh-CN}
\usepackage{xcolor}
\setCJKmainfont{WenQuanYi Zen Hei}
\setCJKfamilyfont{song}{WenQuanYi Zen Hei}

% 调整页面出血
\usepackage[scale=0.75]{geometry}
%\setlength{\hintscolumnwidth}{3cm}           % 如果你希望改变日期栏的宽度

%\usepackage{hyperref}

% 个人信息
\firstname{苏}
\familyname{振兴}
%\title{简历题目 (可选项)}                      % 可选项、如不需要可删除本行
%\address{街道及门牌号}{邮编及城市}             % 可选项、如不需要可删除本行
\mobile{+86~***********}                         % 可选项、如不需要可删除本行
%\phone{+2~(345)~678~901}                          % 可选项、如不需要可删除本行
%\fax{+3~(456)~789~012}                            % 可选项、如不需要可删除本行
\email{suzp1984@gmail.com}                    % 可选项、如不需要可删除本行
\homepage{medium.com/@zpcat}                  % 可选项、如不需要可删除本行
%\extrainfo{附加信息 (可选项)}                  % 可选项、如不需要可删除本行
\photo[64pt][0.4pt]{myself-new}                  % ‘64pt’是图片必须压缩至的高度、‘0.4pt‘是图片边框的宽度 (如不需要可调节至0pt)、’picture‘ 是图片文件的名字;可选项、如不需要可删除本行
%\quote{引言(可选项)}                           % 可选项、如不需要可删除本行

% 显示索引号;仅用于在简历中使用了引言
%\makeatletter
%\renewcommand*{\bibliographyitemlabel}{\@biblabel{\arabic{enumiv}}}
%\makeatother

% 分类索引
%\usepackage{multibib}
%\newcites{book,misc}{{Books},{Others}}
%----------------------------------------------------------------------------------
%            内容
%----------------------------------------------------------------------------------
\begin{document}
%\begin{CJK}{UTF8}{gbsn}                       % 详情参阅CJK文件包
\maketitle

\section{教育背景}
\cventry{2007.8-10.7}{硕士}{中科院金属所}{沈阳} {物理化学硕士学位} {} % {\textit{成绩}}{说明}  % 第3到第6编码可留白
\cventry{2003.8-07.7}{学士}{烟台大学}{烟台} {应用物理学士学位} {} % {\textit{成绩}}{说明}

%\section{毕业论文}
%\cvitem{题目}{\emph{题目}}
%\cvitem{导师}{导师}
%\cvitem{说明}{\small 论文简介}

\section{工作背景}
%\subsection{专业}
\cventry{2017.11-now}{厦门铃盛软件}{移动开发部}{厦门}{}{%
}
\cvitem{职位}{%
\begin{itemize}
\item {\small \emph {安卓软件工程师}}
\end{itemize}
}
\vspace{-4.5mm}
\cvitem{职责}{%
\begin{itemize}
\item {\small \emph {WebRTC安卓客户端开发}}
\item {\small \emph {视频会议系统开发}}
\end{itemize}
}

\cventry{2016.8-17.11}{英特尔(Intel) \& 迈克菲(McAfee)}{手机安全部}{深圳}{}{%
}
\cvitem{职位}{%
\begin{itemize}
\item {\small \emph {安卓软件工程师}}
\end{itemize}
}
\vspace{-4.5mm}
\cvitem{职责}{%
\begin{itemize}
\item {\small \emph {迈克菲手机杀毒软件开发}}
\end{itemize}
}
\vspace{-4.5mm}
\cvitem{说明}{%
\begin{itemize}
\item {\small \emph {McAfee在2017年4月从Intel手机安全部门独立成为Intel持股的独立公司, 这两家公司的工作是连续的。}}
\end{itemize}
}

\cventry{2014.4-16.4}{深圳盒子支付}{研发部}{深圳}{}{%
}
\cvitem{职位}{%
\begin{itemize}
\item {\small \emph {安卓软件工程师}}
\end{itemize}
}
\vspace{-4.5mm}
\cvitem{职责}{%
\begin{itemize}
\item {\small \emph {移动POS机安卓端开发}}
\item {\small \emph {软件架构部门技术调研}}
\end{itemize}
}


\cventry{2010.4-14.4}{深圳鼎智通讯技术有限公司}{智能手机部门}{深圳}{}{%
}
\cvitem{职位}{%
\begin{itemize}
\item {\small \emph {安卓系统工程师}}
\end{itemize}
}
\vspace{-4.5mm}
\cvitem{部门}{%
\begin{itemize}
\item {\small {2010.4 - 2012.1}, \small \emph {深圳鼎智时代}}
\item {\small {2012.2 - 2012.8}, \small \emph {深圳鼎智通讯}}
\item {\small {2012.8 - 2013.8}, \emph {离职}}  
\item {\small {2013.9 - 2014.4}, \small \emph {深圳鼎智时代}}
\end{itemize}
}
\vspace{-4.5mm}
\cvitem{职责}{%
\begin{itemize}
\item {\small \emph {安卓HAL硬件适配层开发维护}}
\item {\small \emph {Marvell, 展讯, 高通, 全志平台开发维护}}
\end{itemize}
}


\section{软件技能}
\cvitem{github}{\small \href{https://github.com/suzp1984}{\color{blue}{https://github.com/suzp1984}}}
\cvitem{blog}{\small \href{https://medium.com/@zpcat}{\color{blue}{https://medium.com/@zpcat}}}
\cvitem{Languages}{\small Kotlin, Java, C/C++, Swift, Emacs Lisp, Python, Javascript, Shell, \LaTeX{}}
\cvitem{OS} {Android, Linux, Mac OS X}
\cvitem{Editor}{Emacs, Android Studio, VS Code}

\section{工作期望}
\vspace{5mm}
\cvitem{}{%
  \begin{itemize}
  \item 期望从事有挑战性的工作,并能够不断提升自己成为一个出色的工程师;
  \item 期望能够参与初创期的项目;
  \item 期望能跟优秀的工程师工作;
\end{itemize}
}

\section{项目经历}
\vspace{5mm}
\subsection{视频会议项目}
\vspace{3mm}
\cventry{2019.1-20.3}{厦门铃盛}{移动开发部}{厦门}{}{%
C++,Mac OS, Android Development.\newline{}%
}
\vspace{-2mm}
\cvitem{描述:}{%
\begin{itemize}
\item 视频会议产品是用来为传统的会议室提供远程音视频的功能,它包括一个PC端(mac)和一个控制端(ipad),其中控制端用来控制会议流程,PC端用来展示音视频和采集音视频。
\end{itemize}
}
\vspace{-4mm}
\cvitem{成绩:}{%
\begin{itemize}
\item 实现了ipad和PC端的直连通讯,因为一般ipad跟PC是部署在同一个内网下的,如果ipad要把控制指令发给PC就需要通过网络推送服务,这有个问题:延时会比较大。通过在PC端实现一个websocket服务器的办法,让ipad跟PC直接连接,解决了延时的问题。并通过每次PC端启动都申请一个随机端口和自签名证书的办法来保证通讯的安全。
\item 内网服务发现方案,上面的websocket服务怎样被ipad发现呢?可以利用IOS/MAC里面的Bonjour方案,在安卓端也有对应叫做网络服务发现(network service discovery)。虽然这个办法没有被采用,我个人是比较喜欢的。
\item Mac OS硬件信息采集,并在Mac端开发一个悬浮窗口用来实时显示硬件信息的变化,方便调试;
\item 会议控制端在安卓平板上面的实现;
\end{itemize}
}
\subsection{TTF Font icon 单元测试}
\vspace{3mm}
\cventry{2018.6-18.7}{厦门铃盛}{移动开发部}{厦门}{}{%
安卓开发\newline{}%
}
\vspace{-2mm}
\cvitem{描述:}{%
\begin{itemize}
\item 用矢量字体来替代小图标是前端常用的办法,目前公司的移动端也在使用,但矢量字体不利于维护,尤其是在多人协作的情况下,当合并代码有冲突的情况下不知道该用那个。
\end{itemize}
}
\vspace{-4mm}
\cvitem{成绩:}{%
\begin{itemize}
\item 通过解析矢量字体的内容,把里面每个字的绘制buffer读取出来并存下它的hash值。这样每次字体文件变化时,运行一个单元测试,会对比每一个项目中用到的字的hash值跟事先存储的hash值之间是否有变化,来确定每个字是否变化了;
\item 用一个gradle插件可以用来重置项目中的比对hash值的存储文件; 如果JVM里面有JavaFX的话也可以用一个插件的命令来查看某个字; 
\item 可以给设计师反馈字体文件中有哪些冗余,就是两个字有相同的内容;
\item 可以给设计师反馈字体文件中有哪些没有用到的字;
\end{itemize}
}

\subsection{安卓Webrtc视频客户端}
\vspace{3mm}
\cventry{2017.12-18.12}{厦门铃盛}{移动开发部}{厦门}{}{%
安卓开发\newline{}%
}
\vspace{-2mm}
\cvitem{描述:}{%
\begin{itemize}
\item 铃盛IM产品有音视频通话功能,其中视频会议采用的是Webrtc SFU架构,我主要工作在安卓客户端业务开发。
\end{itemize}
}
\vspace{-4mm}
\cvitem{成绩:}{%
\begin{itemize}
\item 实现了早期的界面逻辑包括Webrtc集成;
\item 个人比较满意的工作是实现了一个可以上下拖动的布局控件,目前公司产品已经不用了,我又实现了一个类似的放在github \href{https://github.com/suzp1984/AdvancedSlidingPanel}{\color{blue}{Advanced Sliding Panel}}
\item 基于我对webrtc sdk的了解,希望能够继续从事更加深入的工作;包括安卓端视频渲染的问题,相对是个难点,个人也想有所突破;
\end{itemize}
}
\subsection{McAfee 安卓杀毒软件}
\vspace{3mm}
\cventry{2016.8-17.7}{英特尔 \& 迈克菲}{手机安全部}{深圳}{}{%
安卓开发\newline{}%
}
\vspace{-2mm}
\cvitem{描述:}{%
\begin{itemize}
\item McAfee 安卓杀毒软件是用来检查安卓手机安全隐患的应用.
\end{itemize}
}
\vspace{-4mm}
\cvitem{成绩:}{%
\begin{itemize}
\item 安卓界面开发,用安卓material design的原则重构页面;
\end{itemize}
}

\subsection{盒子支付钱盒项目}
\vspace{3mm}
\cventry{2014.7-15.5}{盒子支付有限公司}{软件部}{深圳}{}{%
安卓开发 \newline{}%
}
\vspace{-2mm}
\cvitem{描述:}{%
\begin{itemize}
\item 钱盒作为盒子支付的核心业务,本质是一个移动端的POS收款机器,它由手机端的应用,后台交易系统和智能终端设备盒子组成。首先,终端盒子要通过音频口或者蓝牙的方式连接到手机端,客户通过在盒子端刷银行卡来同商户完成一笔交易。
\item 智能盒子硬件部分可通过同手机通讯方式的不同分为音频口通讯,蓝牙通讯和UART串口通讯的方式,又根据支持的银行卡类型的不同分为磁条卡和IC卡支付。
\end{itemize}
}
\vspace{-4mm}
\cvitem{成绩:}{%
\begin{itemize}
\item 将主要的业务对象独立到单独的安卓模块中,并能在其它的项目中重用此模块。
\item 按照传统的软件设计流程:添加单元测试,写软件设计文档,并维护一份软件编程规范。保证软件开发可持续和可维护。
\item 在对象的设计中比较有价值的是抽象出同硬件的通信类型的接口,使得音频和蓝牙两种通信方案相互独立,并能够在以后的项目中引入串口通讯时,可以方便的重用所有的代码。
\item 使用JNI移植jbig图片编码技术到java平台, JBIG是一种单颜色空间的无损压缩格式,常用于能提供压力触摸笔签名的银联POS机。
\item 因为Jbig的工作是帮助后端的项目做的,公司安卓平台没有使用,现把安卓平台的demo放到github: \href{https://github.com/suzp1984/jbig-android}{\color{blue}{Jbig-Android}}
\end{itemize}}
\cvitem{}{}

\subsection{盒子支付架构组工作}
\vspace{3mm}
\cventry{2015.7-16.3}{盒子支付}{软件部}{深圳}{}{%
软件工程师\newline{}%
}
\vspace{-2mm}
\cvitem{描述:}{%
\begin{itemize}
\item Android应用架构研究
\item Git/Gitlab 技术培训
\item Apache/Nginx 日志分析,生成静态报告
\item OpenResty 应用开发
\item Docker以及自动配置系统方案研究
\end{itemize}
}
\vspace{-4mm}
\cvitem{成绩:}{%
\begin{itemize}
\item 在公司推广了gitlab和git的使用,抛弃了以往的svn版本管理工具。
\item android MVP/MVC 架构实践
\item 检查Servlet session是否合法的OpenResty应用
\end{itemize}
}

\cvitem{}{}
\subsection{蓝牙低功耗可穿戴设备项目}
\vspace{3mm}
\cventry{2013.10-14.3}{鼎智时代}{软件部}{深圳}{}{%
安卓开发\newline{}%
}
\vspace{-2mm}
\cvitem{描述:}{%
\begin{itemize}
\item 研究BroadCom 的BLE方案在可穿戴领域的应用
\end{itemize}}
\vspace{-4mm}
\cvitem{成绩:}{%
\begin{itemize}
\item 开源了BLE在安卓端的调试工具 \href{https://github.com/suzp1984/Light_BLE}{\color{blue}{Light-BLE}}
\end{itemize}
}
\cvitem{}{}

\subsection{展讯安卓平台工厂自动测试工具}
\vspace{3mm}
\cventry{2012.5-12.7}{鼎智通讯}{软件部}{深圳}{}{%
安卓rom开发\newline{}%
}
\vspace{-2mm}
\cvitem{描述:}{%
\begin{itemize}
\item 展讯平台的pcb板出厂测试模块,利用安卓系统的C层的代码,将所有外设都在一个出厂夹具中自动跑一遍,然后测试结果会显示在夹具的LCD屏上。
\end{itemize}
}
\vspace{-4mm}
\cvitem{成绩:}{%
\begin{itemize}
\item 方便工厂在贴片完成之后就能迅速将有问题的pcb板检查出来,提高手机的良品率。
\end{itemize}
}
\cvitem{}{}

\subsection{安卓系统开发}
\vspace{3mm}
\cventry{2010.5-12.8}{鼎智时代,鼎智通讯}{软件部}{深圳}{}{安卓rom开发维护\newline{}}
\vspace{-2mm}
\cvitem{描述:}{%
\begin{itemize}
\item 从android1.6开始到4.3,在鼎智做android 方案开发,经历过Marvell, 展讯,高通,全志平台的方案。
\item 负责makefile, HAL硬件适配层的开发。
\item 负责蓝牙,wifi模块调试。
\end{itemize}}


%\cvdoubleitem{类别 1}{XXX, YYY, ZZZ}{类别 4}{XXX, YYY, ZZZ}
%\cvdoubleitem{类别 2}{XXX, YYY, ZZZ}{类别 5}{XXX, YYY, ZZZ}
%\cvdoubleitem{类别 3}{XXX, YYY, ZZZ}{类别 6}{XXX, YYY, ZZZ}

%\section{个人兴趣}
%\cvitem{骑车}{\small 喜欢骑车旅行}
%\cvitem{爱好 2}{\small 说明}
%\cvitem{爱好 3}{\small 说明}

%\section{其他 1}
%\cvlistitem{项目 1}
%\cvlistitem{项目 2}
%\cvlistitem{项目 3}

\renewcommand{\listitemsymbol}{-}             % 改变列表符号

%\section{其他 2}
%\cvlistdoubleitem{项目 1}{项目 4}
%\cvlistdoubleitem{项目 2}{项目 5\cite{book1}}
%\cvlistdoubleitem{项目 3}{}

% 来自BibTeX文件但不使用multibib包的出版物
%\renewcommand*{\bibliographyitemlabel}{\@biblabel{\arabic{enumiv}}}% BibTeX的数字标签
\nocite{*}
\bibliographystyle{plain}
\bibliography{publications}                    % 'publications' 是BibTeX文件的文件名

% 来自BibTeX文件并使用multibib包的出版物
%\section{出版物}
%\nocitebook{book1,book2}
%\bibliographystylebook{plain}
%\bibliographybook{publications}               % 'publications' 是BibTeX文件的文件名
%\nocitemisc{misc1,misc2,misc3}
%\bibliographystylemisc{plain}
%\bibliographymisc{publications}               % 'publications' 是BibTeX文件的文件名

%\clearpage\end{CJK}
\end{document}


%% 文件结尾 `template-zh.tex'.
