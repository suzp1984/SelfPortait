%%%%%%%%%%%%%%%%%%%%%%%%%%%%%%%%%%%%%%%%%%%%%%%%%%%%%%%%%%%%
% Name: XeTeX+xeCJK日常使用模板
% Author: Lox Freeman
% Email: xiaohanyu1988@gmail.com
% Homepage: http://cnlox.is-programmer.com
% 
% 本文档可以自由转载、修改,希望能给广大TeXer的中文之路提供一些方便。
%%%%%%%%%%%%%%%%%%%%%%%%%%%%%%%%%%%%%%%%%%%%%%%%%%%%%%%%%%%%

\documentclass[a4paper, 10pt, titlepage]{article}
% 设定纸张大小为A4, 基本字体大小为12pt, 文章题目单独为一页, 
% 文档类型为article

%%%%%%%%%%%%%%%%%%%%%%%%%xeCJK相关宏包%%%%%%%%%%%%%%%%%%%%%%%%%
%\usepackage{xltxtra,fontspec,xunicode}
%\usepackage[slantfont, boldfont, CJKaddspaces]{xeCJK} 
% \CJKsetecglue{\hskip 0.15em plus 0.05em minus 0.05em}
% slanfont: 允许斜体
% boldfont: 允许粗体
% CJKnormalspaces: 仅忽略汉字之间的空白,但保留中英文之间的空白。 
% CJKchecksingle: 避免单个汉字单独占一行。
% CJKaddspaces: [备选]忽略汉字之间的空白,并且自动在中英文转换时插入空白。

% character encoding
\usepackage{xunicode, xltxtra}
%\setmainfont[Mapping=tex-text]{文泉驿正黑}
%\setsansfont[Mapping=tex-text]{文泉驿正黑}
\XeTeXlinebreaklocale "zh"             % 针对中文进行断行
\XeTeXlinebreakskip = 0pt plus 1pt minus 0.1pt
                                       % 给予TeX断行一定自由度
%%%%%%%%%%%%%%%%%%%%%%%%%xeCJK%%%%%%%%%%%%%%%%%%%%%%%%%%%%%%%%

%%%%%%%%%%%%%日常所用宏包、通通放在一起%%%%%%%%%%%%%%%%%%%%%%%%%%%%
% 什么常用的宏包都可以放这里。下面是我常用的宏包,每个都给出了简要注释
\usepackage[top=2.2cm, bottom=2cm, left=2cm, right=2cm]{geometry}                               
                                     % 控制页边距
\usepackage{enumerate}               % 控制项目列表
\usepackage{multicol}                % 多栏显示

\usepackage[%
    pdfstartview=FitH,%
    %CJKbookmarks=true,%
    bookmarks=true,%
    bookmarksnumbered=true,%
    bookmarksopen=true,%
    colorlinks=true,%
    citecolor=blue,%
    linkcolor=blue,%
    anchorcolor=green,%
    urlcolor=blue%
]{hyperref}                          % hyperref宏包,生成可定位点击的超链接


\usepackage{titlesec}                % 控制标题
\usepackage{booktabs}                % 控制表格样式
\usepackage{titletoc}                % 控制目录
\usepackage{type1cm}                 % 控制字体大小
\usepackage{indentfirst}             % 首行缩进,用\noindent取消某段缩进
\usepackage{color,xcolor}            % 支持彩色文本、底色、文本框等
\usepackage{amsmath}                 % AMS LaTeX宏包
% \usepackage{amssymb}
% \usepackage{bbding}                % 一些特殊符号
% \usepackage{cite}                  % 支持引用
% \usepackage{latexsym}              % LaTeX一些特殊符号宏包
% \usepackage{bm}                    % 数学公式中的黑斜体
% \usepackage{relsize}               % 调整公式字体大小:\mathsmaller, \mathlarger
% \makeindex                         % 生成索引

%%%%%%%%%%%%%%%%%%%%%%%%%基本插图方法%%%%%%%%%%%%%%%%%%%%%%%%%%%
\usepackage{graphicx}                % 图形宏包
\usepackage{subfig}                  % 多个图形并排,参加lnotes.pdf
% \begin{figure}[htbp]               % 控制插图位置
%   \setlength{\abovecaptionskip}{0pt}    
%   \setlength{\belowcaptionskip}{10pt}
                                     % 控制图形和上下文的距离
%   \centering                       % 使图形居中显示
%   \includegraphics[width=0.8\textwidth]{CTeXLive2008.jpg}
                                     % 控制图形显示宽度为0.8\textwidth
%   \caption{CTeXLive2008安装过程} \label{fig:CTeXLive2008}
                                     % 图形题目和交叉引用标签
% \end{figure}
%%%%%%%%%%%%%%%%%%%%%%%%%插图方法结束%%%%%%%%%%%%%%%%%%%%%%%%%%%

%%%%%%%%%%%%%%%%%%%%pgf/tikz绘图宏包设置%%%%%%%%%%%%%%%%%%%%
\usepackage{pgf,tikz}
\usetikzlibrary{shapes,automata,snakes,backgrounds,arrows}
\usetikzlibrary{mindmap} 
% \usepackage[shell,pgf,outputdir={docgraphs/}]{dot2texi}
                                     % 直接在latex文档中使用graphviz/dot语言
                                     % 也可以用dot2tex工具将dot文件转换成tex文件再include进来
%%%%%%%%%%%%%%%%%%%%pgf/tikz end%%%%%%%%%%%%%%%%%%%%


%%%%%%%%%%%%%%%%%%%%%%%%%%fancyhdr设置页眉页脚%%%%%%%%%%%%%%%%%%%%
\usepackage{fancyhdr}                % 页眉页脚
\pagestyle{plain}                    % 页眉页脚风格
\setlength{\headheight}{15pt}        % 有时会出现\headheight too small的warning
%\fancyhf{}                          % 清空当前页眉页脚的默认设置
%%%%%%%%%%%%%%%%%%%%%%%%%fancyhdr设置结束%%%%%%%%%%%%%%%%%%%%%%%


%%%%%%%%%%%%%%%%%%%%%%%%%listings宏包粘贴源码%%%%%%%%%%%%%%%%%%%%
\usepackage{listings}                % 方便粘贴源代码,部分代码高亮功能
\lstloadlanguages{}                  % 所要粘贴代码的编程语言

%%%%设置listings宏包的一些全局样式%%%%
%%%参见http://hi.baidu.com/shawpinlee/blog/item/9ec431cbae28e41cbe09e6e4.html%%%%
\lstset{
showstringspaces=false               % 设定是否显示代码之间的空格符号
numbers=left,                        % 在左边显示行号
numberstyle=\tiny,                   % 设定行号字体的大小
basicstyle=\tiny,                    % 设定字体大小\tiny, \small, \Large等等
keywordstyle=\color{blue!70}, commentstyle=\color{red!50!green!50!blue!50},
                                     % 关键字高亮
frame=shadowbox,                     % 给代码加框
rulesepcolor=\color{red!20!green!20!blue!20},
escapechar=`,                        % 中文逃逸字符,用于中英混排
xleftmargin=2em,xrightmargin=2em, aboveskip=1em,
breaklines,                          % 这条命令可以让LaTeX自动将长的代码行换行排版
extendedchars=false                  % 这一条命令可以解决代码跨页时,章节标题,页眉等汉字不显示的问题
}
%%%%%%%%%%%%%%%%%%%%%%%%%listings宏包设置结束%%%%%%%%%%%%%%%%%%%%

%%%%%%%%%%%%%%%%%%%%%%%%%附录设置%%%%%%%%%%%%%%%%%%%%%%%%%
\usepackage[title,titletoc,header]{appendix}
%%%%%%%%%%%%%%%%%%%%%%%%%附录设置结束%%%%%%%%%%%%%%%%%%%%%%%%%

%%%%%%%%%%%%%%%%%%%%%%%%%xeCJK字体设置%%%%%%%%%%%%%%%%%%%%%%%%%
%\punctstyle{kaiming}                                        % 设置中文标点样式
                                                            % 支持quanjiao、banjiao、kaiming等多种方式
%\setCJKmainfont[BoldFont=Adobe Heiti Std]{Adobe Song Std}   % 设置缺省中文字体
%\setCJKsansfont[BoldFont=Adobe Heiti Std]{Adobe Kaiti Std}  % 设置中文无衬线字体
%\setCJKmonofont{Adobe Heiti Std}                            % 设置等宽字体
%\setmainfont{DejaVu Serif}                                  % 英文衬线字体
%\setmonofont{DejaVu Sans Mono}                              % 英文等宽字体
%\setsansfont{DejaVu Sans}                                   % 英文无衬线字体

%%%%定义新字体%%%%
%\setCJKfamilyfont{song}{Adobe Song Std}                     
%\setCJKfamilyfont{kai}{Adobe Kaiti Std}
%\setCJKfamilyfont{hei}{Adobe Heiti Std}
%\setCJKfamilyfont{fangsong}{Adobe Fangsong Std}
%\setCJKfamilyfont{lisu}{LiSu}
%\setCJKfamilyfont{youyuan}{YouYuan}

%\newcommand{\song}{\CJKfamily{song}}                       % 自定义宋体
%\newcommand{\kai}{\CJKfamily{kai}}                         % 自定义楷体
%\newcommand{\hei}{\CJKfamily{hei}}                         % 自定义黑体
%\newcommand{\fangsong}{\CJKfamily{fangsong}}               % 自定义仿宋体
%\newcommand{\lisu}{\CJKfamily{lisu}}                       % 自定义隶书
%\newcommand{\youyuan}{\CJKfamily{youyuan}}                 % 自定义幼圆

\newcommand{\yihao}{\fontsize{26pt}{36pt}\selectfont}       % 一号, 1.4倍行距
\newcommand{\erhao}{\fontsize{22pt}{28pt}\selectfont}       % 二号, 1.25倍行距
\newcommand{\xiaoer}{\fontsize{18pt}{18pt}\selectfont}      % 小二, 单倍行距
\newcommand{\sanhao}{\fontsize{16pt}{24pt}\selectfont}      % 三号, 1.5倍行距
\newcommand{\xiaosan}{\fontsize{15pt}{22pt}\selectfont}     % 小三, 1.5倍行距
\newcommand{\sihao}{\fontsize{14pt}{21pt}\selectfont}       % 四号, 1.5倍行距
\newcommand{\bansi}{\fontsize{13pt}{19.5pt}\selectfont}     % 半四, 1.5倍行距
\newcommand{\xiaosi}{\fontsize{12pt}{18pt}\selectfont}      % 小四, 1.5倍行距
\newcommand{\dawu}{\fontsize{11pt}{11pt}\selectfont}        % 大五, 单倍行距
\newcommand{\wuhao}{\fontsize{10.5pt}{10.5pt}\selectfont}   % 五号, 单倍行距
%%%%%%%%%%%%%%%%%%%%%%%%%xeCJK字体设置结束%%%%%%%%%%%%%%%%%%%%%%

%%%%%%%%%%%%%%%%%%%%%%%%%一些关于中文文档的重定义%%%%%%%%%%%%%%%%%

%%%%%数学公式定理的重定义%%%%
\newtheorem{example}{例}                                   % 整体编号
\newtheorem{algorithm}{算法}
\newtheorem{theorem}{定理}[section]                         % 按 section 编号
\newtheorem{definition}{定义}
\newtheorem{axiom}{公理}
\newtheorem{property}{性质}
\newtheorem{proposition}{命题}
\newtheorem{lemma}{引理}
\newtheorem{corollary}{推论}
\newtheorem{remark}{注解}
\newtheorem{condition}{条件}
\newtheorem{conclusion}{结论}
\newtheorem{assumption}{假设}

%%%%章节等名称重定义%%%%
\renewcommand{\contentsname}{目录}     
\renewcommand{\abstractname}{摘要}
\renewcommand{\indexname}{索引}
\renewcommand{\listfigurename}{插图目录}
\renewcommand{\listtablename}{表格目录}
\renewcommand{\figurename}{图}
\renewcommand{\tablename}{表}
\renewcommand{\appendixname}{附录}
\renewcommand{\appendixpagename}{附录}
\renewcommand{\appendixtocname}{附录}
\renewcommand\refname{参考文献} 

%%%%设置chapter、section与subsection的格式%%%%
\titleformat{\chapter}{\centering\huge}{第\thechapter{}篇}{1em}{\textbf}
\titleformat{\section}{\centering\sihao}{\thesection}{1em}{\textbf}
\titleformat{\subsection}{\xiaosi}{\thesubsection}{1em}{\textbf}
\titleformat{\subsubsection}{\xiaosi}{\thesubsubsection}{1em}{\textbf}

%%%%%%%%%%%%%%%%%%%%%%%%%中文重定义结束%%%%%%%%%%%%%%%%%%%%

%%%%%%%%%%%%%%%%%%%%%%%%%一些个性设置%%%%%%%%%%%%%%%%%%%%%%
% \renewcommand{\baselinestretch}{1.3}     % 效果同\linespread{1.3}
% \pagenumbering{arabic}                   % 设定页码方式,包括arabic、roman等方式
% \sloppy                                  % 有时LaTeX无从断行,产生overfull的错误,
                                           % 这条命令降低LaTeX断行标准
% \setlength{\parskip}{0.5\baselineskip}     % 设定段间距
\linespread{1}                             % 设定行距
\newcommand{\pozhehao}{\kern0.3ex\rule[0.8ex]{2em}{0.1ex}\kern0.3ex}
                                           % 中文破折号,据说来自清华模板

\usepackage{textcomp}             % for \textcelsius
\renewcommand{\arraystretch}{1.5} % 將表格行間距加大為原來的 1.5 倍
%%%%%%%%%%%%%%%%%%%%%%%%%个性设置结束%%%%%%%%%%%%%%%%%%%%%%

%%%%%%%%%%%%%%%%%%%%%%%%%bibtex设置%%%%%%%%%%%%%%%%%%%%%%%%%
\bibliographystyle{plain}                  % 设定参考文献显示风格
%%%%%%%%%%%%%%%%%%%%%%%%%bibtex设置结束%%%%%%%%%%%%%%%%%%%%%%%%%

%%%%%%%%%%%%%%%%%%%%%%%%%正文部分%%%%%%%%%%%%%%%%%%%%%%%%%
\begin{document}
\renewcommand{\normalsize}{\wuhao}         % 设定正文字体大小

\setlength{\parindent}{0em}                    
% 设定首行缩进为2em。注意此设置一定要在document环境之中。
% 这可能与\setlength作用范围相关
\newcommand{\mysection}[1]{\vspace{5pt} {\bfseries \textsl{#1}} \\ {\color{gray} \rule[5pt]{\textwidth}{0.3pt}}}
\renewcommand{\labelitemi}{$\bullet$}

\definecolor{headings}{HTML}{701112}  % dark red
\newcommand{\cvtitle}[1]{\centerline{\huge \textbf{#1}} \bigskip}
%\newcommand{\career}[2]{\vspace{5pt} {{\bfseries \textsl{#1}} {\normalsize{#2}}} \\ {\color{gray} \rule[5pt]{\textwidth}{0.3pt}}}
\newcommand{\career}[2]{\vspace{5pt} {{\bfseries \textsl{#1}} \hspace{5pt} {\normalsize{#2}}} \\}
\pagestyle{empty}

\cvtitle{Su ZhenXing}
%\career{OBJECTIVE}{C++/C Software Engineer}
\career{CAREER}{Linux Software Engineer}

\mysection{INFORMATION}
\begin{minipage}[t]{0.495\textwidth}
  Native Name: ZhenXing Su \\
  Gender: Male \\
  Date of Birth: October 7th, 1984\\
  Hometown: ShanDong Province
\end{minipage}
\begin{minipage}[t]{0.495\textwidth}
  Years of Working: 4+ years\\
  Phone: (+86) 15012462201 \\
  %Twitter: \href{https://twitter.com/mutse\_young}{https://twitter.com/mutse\_young} \\
  Email: \href{mailto:suzp1984@gmail.com}{suzp1984@gmail.com} \\
  Blog: \href{http://zpcat.blogspot.com}{http://zpcat.blogspot.com}
\end{minipage}

\vspace{3mm}
\mysection{EDUCATION}

\begin{itemize}

\item Institute of Metal Research, Chinese Academy of Sciences \hfill \textrm{ShenYang, LiaoNing provice}
  \begin{itemize}
    \item Master degree in Physics. \hfill \textrm{Sep 2007-Jun 2010}
  \end{itemize}
\end{itemize}

\begin{itemize}

\item YanTai University of Technology \hfill \textrm{YanTai, ShanDong Province}
  \begin{itemize}
  \item Bachelor degree in Physics.  \hfill \textrm{Sep 2003-Jun 2007}
  \end{itemize}

\end{itemize}

\mysection{SKILLS}
\begin{itemize}
\item \textbf{github:}{\small \href{https://github.com/suzp1984}{\color{blue}{https://github.com/suzp1984}}}
\item \textbf{Blog:}{\small \href{http://zpcat.blogspot.com/}{\color{blue}{http://zpcat.blogspot.com/}}}
\item \textbf{Programming Languages:} C(expert), Emacs Lisp(proficient), Bash shell(proficient), Python(proficient), Java(prior experience), node.js(posterior experience), C++(posterior experience)
\item \textbf{Operation Systems:} Linux(Ubuntu, Fedora), Mac OS X
\item \textbf{Editor:} Emacs(expert), Intellij(Java IDE), Qt-Creator(Qt)
\item \textbf{Miscellaneous:} {\LaTeX}, etc.
\end{itemize}

\mysection{PROJECTS EXPERIENCE}
\begin{itemize}

\item \textbf{Android building system development}
  \begin{itemize}
    \item rebuild the Android build system to fulfill the target of stronger customization, developed by Bash Shell, Python.
  \end{itemize}

\item \textbf{porting Wifi and Bluetooth module}
  \begin{itemize}
    \item Porting and maintance the Wifi and Bluetooth module, and familiar with wpa\_supplicant and bluez stack.
  \end{itemize}

\item \textbf{Android HAL layer developement}
  \begin{itemize}
    \item android HAL layer development, mainly C source code, the adapter layers for sensors, GPS.
  \end{itemize}

\item \textbf{Kernel configuration tools}
  \begin{itemize}
    \item Kernel configuration customization tools development, developed by C, there are an user client to generate a block of data(firmware), and a kernel part to parser that data block when it startup.
  \end{itemize}

\item \textbf{Factory Test toolkits}
  \begin{itemize}
    \item a. factory test tool for ordinary customers, running in native C level. \newline{}
    \item b. Auto Test Daemon who listening the commands from the UART. 
  \end{itemize}

\item \textbf{Configuration Generation tool}
  \begin{itemize}
    \item Writen by python to parser multi format of configuration files, and generate the finally format.
  \end{itemize}

\item \textbf{Bluetooth Low engergy research}
  \begin{itemize}
    \item Pre-research of wearable devices, famillar with bluetooth low enegery protocol.
  \end{itemize}

\item \textbf{Android wifi/bt and Recovery modules stress test tool}

\end{itemize}


\mysection{WORK EXPERIENCE}

\begin{itemize}

\item \textbf{Shenzhen topwise Technology} \hfill \textrm{August 2010-Now}
  \begin{itemize}
    \item Android system development
  \end{itemize}

\end{itemize}

\mysection{OPEN SOURCE PROJECTS}

\begin{itemize}

\item {\color{blue}{TTS-mode}} TTS(Text to speech) mode for emacs
  \begin{description}
    \item Git Repo: \url{https://github.com/suzp1984/tts-mode}
  \end{description}

\item {\color{blue}{fs-mode}} Linux file system management mode for emacs
  \begin{description}
    \item Git Repo: \url{https://github.com/suzp1984/fs-mode}
  \end{description}

\item {\color{blue}{pulseaudio-mode}}
  \begin{description}
    \item Git Repo: \url{https://github.com/suzp1984/pulseaudio-mode}
  \end{description}

\item {\color{blue}{genesis}} a plugin desgined middle ware, implemented the reactor module
  \begin{description}
    \item Git Repo: \url{https://github.com/suzp1984/genesis}
  \end{description}
\item {\color{blue}{pyqt5-book-code}} Porting the source codes of book, Rapid GUI Programming with python and Qt, from PyQt4 to PyQt5
  \begin{description}
    \item Git Repo: \url{https://github.com/suzp1984/pyqt5-book-code}
  \end{description}

\end{itemize}

\mysection{INTERESTS}

\begin{itemize}
\item Reading \& Blogging.
\item Coding, Linux \& OpenSource.
\end{itemize}

%\mysection{SELF ASSESSMENT}

%\begin{itemize}
%  \item \textsc{Ability to learn} Independent thinking, with strong self-learning ability.
%  \item \textsc{Teamwork} \hfill Have team spirit, and practical in work.
%  \item \textsc{Character} Possessed of cheerful easy-going personality, and treat people sincerely.
%  \item \textsc{Others} With strong mental endurance.
%\end{itemize}

%\begin{tabular}{ll}
% 	\textsc{Self-learning} & Independent thinking, with strong self-learning ability.\\
%	\textsc{Teamwork} & Have team spirit, and practical in work.\\
%\end{tabular}

\end{document}

